\problemname{Cent Savings}

\illustration{0.5}{dividerbar}{\href{https://commons.wikimedia.org/wiki/File:Customer_divider_bar.jpg}{Picture} by Tijmen Stam via Wikimedia Commons, cc by-sa}%
\noindent
To host a regional contest like NWERC a lot of preparation is necessary:
organizing rooms and computers, making a good problem set, inviting
contestants, designing T-shirts, booking hotel rooms and so on. I am
responsible for going shopping in the supermarket.

When I get to the cash register, I put all my $n$ items on the conveyor belt
and wait until all the other customers in the queue in front of me are served.
While waiting, I realize that this supermarket recently started to round the
total price of a purchase to the nearest multiple of $10$ cents (with $5$ cents
being rounded upwards). For example, $94$ cents are rounded to $90$ cents,
while $95$ are rounded to $100$.

It is possible to divide my purchase into groups and to pay for the parts
separately. I managed to find $d$ dividers to divide my purchase in up to $d+1$
groups. I wonder where to place the dividers to minimize the total cost of
my purchase.  As I am running out of time, I do not want to rearrange items on
the belt.

\section*{Input}

The input consists of:
\begin{itemize}
   \item one line with two integers $n$ ($1 \leq n \le 2\,000$) and $d$ ($1 \leq
d \le 20$), the
number of items and the number of available dividers;
   \item one line with $n$ integers $p_1, \ldots p_n$ ($1 \leq p_i \leq 10\,000$ for $1 \le i \le n$), the prices of
	   the items in cents. The prices are given in the same order as the items appear on the belt.
\end{itemize}

\section*{Output}

Output the minimum amount of money needed to buy all the items, using up to $d$ dividers.
