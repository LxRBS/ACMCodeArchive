\problemname{Indoorienteering}

% \illustration{.4}{orienteering.jpg}{Two runners approaching a control point. Photo by % Flickr user oskar karlin}
%   \href{https://www.flickr.com/photos/oskarlin/9419772165}{Oskar Karlin}.}%
\illustration{.35}{fredriktommy}{Fredrik and Tommy lost in the B building. Photo by Tommy Olsson. cc-by-sa}%

\newcommand{\Lukas}{Luk\'a\v{s}}%
\noindent
\Lukas{} really likes orienteering, a sport that requires locating control points in rough terrain. To entertain the NWERC participants \Lukas{} wants to organize an orienteering race. However, it would be too harsh for the participants to be outdoors in this cold Swedish November weather, so he decided to jump on the new trend of indoor races, and set the race inside the B building of Linköping University.

\Lukas{} has already decided on the locations of the control points. He has also decided on the exact length of the race, so the only thing remaining is to decide in which order the control points should be visited such that the length of the total race is as he wishes. Because this is not always possible, he asks you to write a program to help him.

\emph{Note from the organizer: the NWERC indoorienteering race for this year has been cancelled since we neglected to apply for an orienteering permit in time from the university administration.  (We still need you to solve the problem so that we can organize it for next year.)}

% Second option
% \emph{Note from the organizer: good work, you are now at the second last checkpoint and the race is almost finished!  See you in 5 hours at the last checkpoint!}

\section*{Input}

The input consists of:
\begin{itemize}
   \item one line with two integers $n$ ($2\leq n \leq 14$) and $L$ ($1 \leq L \leq 10^{15}$), the number of control points and the desired length of the race, respectively;
   \item $n$ lines with $n$ integers each.  The $j$th integer on the $i$th line, $d_{ij}$, denotes the distance between control point $i$ and $j$  ($1\leq d_{ij} \leq L$ for $i \not= j$, and $d_{ii} = 0$). For all $1 \leq i,j,k \leq N$ it is the case that $d_{ij} = d_{ji}$ and $d_{ij} \leq d_{ik} + d_{kj}$.
\end{itemize}

\section*{Output}

Output one line with ``\texttt{possible}'' if it is possible to visit all control points once in some order and directly return to the first one such that the total distance is exactly~$L$, and ``\texttt{impossible}'' otherwise.
