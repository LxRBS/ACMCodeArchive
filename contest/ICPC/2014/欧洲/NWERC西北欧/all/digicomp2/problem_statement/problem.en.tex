\problemname{Digi Comp II}

\illustration{0.5}{digicomp2}{Photo by \href{https://www.flickr.com/photos/oskay/}{oskay} from Flickr, cc by-sa}%
\noindent
The Digi Comp II is a machine where balls enter from the
top and find their way to the bottom via a certain circuit defined by
switches.  
Whenever a ball falls on a switch it either goes to the left or to the
right depending on the state of the switch and flips this
state in the process.
Abstractly it can be modelled by a directed graph with a vertex of
outdegree $2$ for each switch and in addition a designated end vertex
of outdegree $0$.  One of the switch vertices is the start vertex, it
has indegree $0$.  Each switch vertex has an internal state (L/R). A
ball starts at the start vertex and follows
a path down to the end vertex, where at each switch vertex it will
pick the left or right outgoing edge based on the internal state of
the switch vertex. The internal state of a vertex is flipped after a ball passes through.
A ball always goes down and therefore cannot get into a loop. 

One can ``program'' this machine by specifying the graph structure, the
initial states of each switch vertex and the number of balls that enter. The result 
of the computation is the state of the switches at the end of the
computation. Interestingly one can program quite sophisticated
algorithms such as addition, multiplication, 
division and even the stable marriage problem. However, it is not
Turing complete. 

\section*{Input}

The input consists of:
\begin{itemize}
\item one line with two integers $n$ ($0\le n\le10^{18}$) and
  $m$ ($1\le m\le 500\,000$), the number of balls and the number of switches of the graph;
\item $m$ lines describing switches $1$ to $m$ in order. Each line
  consists of a single character $c$ (`\texttt{L}' or `\texttt{R}')
  and two integers $L$ and $R$ ($0\le L,R\le m$), describing the
  initial state ($c$) of the switch 
and the destination vertex of the left ($L$) and right ($R$) outgoing
edges. $L$ and $R$ can be equal.
\end{itemize}
Vertex number $0$ is the end vertex and vertex $1$ is the
start vertex. There are no loops in the graph, i.e., after going
through a switch a ball can never return to it.

\section*{Output}

Output one line with a string of length $m$ consisting of the characters `\texttt{L}' and `\texttt{R}', describing the final state of the switches ($1$ to $m$ in order). 
