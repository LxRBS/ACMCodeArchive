\begin{problem}{Lost Permutation}{standard input}{standard output}{2 seconds}{512 megabytes}

\textit{This is an interactive problem.}

You once had a permutation $\pi$ of size $n$. And now it's gone. All you have left is an old device you made while studying group theory. To try and recover $\pi$ you can input a permutation $f$ of size $n$ into this device. This device will then display a permutation $\pi^{-1} \circ f \circ \pi$. Find $\pi$ using at most two interactions with the device.


A permutation of size $n$ is a sequence of $n$ distinct integers from $1$ to $n$. The \textit{composition} of two permutations $a$ and $b$ is a permutation $a \circ b$ such that $(a \circ b)_i = b_{a_i}$. That is, if we consider a permutation as an action on $n$ elements, moving element at position $i$ to $a_i$, then $a \circ b$ is the action that applies $a$, then applies $b$, so that element at position $i$ first moves to $a_i$, then moves to $b_{a_i}$. Note that some definitions of composition use the reverse order.

The inverse permutation $\pi^{-1}$ is a permutation $\sigma$ such that $\sigma_{\pi_i} = i$. The composition of a permutation and its inverse is equal to an identity permutation: $(\pi \circ \pi^{-1})_i =  (\pi^{-1} \circ \pi)_i = i$ for all $i$ from $1$ to $n$. For example, if $a = (4, 1, 3, 2)$ and $b = (3, 2, 1, 4)$, then $a \circ b = (4, 3, 1, 2)$, $a^{-1} = (2, 4, 3, 1)$ and $a^{-1} \circ b \circ a = (1, 2, 4, 3)$.

\Interaction
Your program has to process multiple test cases in a single run. First, the testing system writes $t$, the number of test cases ($t \ge 1$). Then, $t$ test cases should be processed one by one.

In each test case your program should start by reading the integer $n$ ($3 \le n \le 10^4$), the size of permutation $\pi$. The sum of $n$ over all test cases does not exceed $10^4$. Then, your program can make queries of two types:

\vspace{-1mm}
\begin{shortitems}
\item \t{? $f_1 \, f_2 \, \ldots \, f_n$}, values $f_1, f_2, \ldots, f_n$ form a permutation of $1, 2, \ldots, n$. The testing system responds with a permutation $g_1, g_2, \ldots, g_n$, where $g = \pi^{-1} \circ f \circ \pi$.

\item \t{! $\pi_1 \, \pi_2 \, \ldots \, \pi_n$}~--- your guess for the secret permutation.
\end{shortitems}
\vspace{-1mm}

You can use at most two queries of the first type in each test case. After your program makes a query of the second type, it should continue to the next test case (or exit if that test case was the last one). 

\Example

\begin{example}
\exmpfile{example.01}{example.01.a}%
\end{example}

\Note
There are two test cases in the first test. In the first test case, $\pi = (4, 1, 3, 2)$ is the only permutation that satisfies $\pi^{-1} \circ (3, 2, 1, 4) \circ \pi = (1, 2, 4, 3)$ and $\pi^{-1} \circ (2, 4, 3, 1) \circ \pi = (2, 4, 3, 1)$. In the second test case, based on the interaction, $\pi$ can be equal to either $(1, 3, 2)$, $(2, 1, 3)$, or $(3, 2, 1)$. The solution got lucky and guessed the correct one: $(3, 2, 1)$.

\end{problem}

