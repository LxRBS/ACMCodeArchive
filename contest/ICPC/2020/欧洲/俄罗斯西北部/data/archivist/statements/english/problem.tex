\begin{problem}{Archivist}{standard input}{standard output}{2 seconds}{512 megabytes}

The team of problemsetters of Northwestern Russia Regional Contest welcomes you! Our regional contest was founded in 1995 under the name ``Collegiate Programming Championship of St Petersburg''. Here is the list of the contest winners:

\begin{shortitems}
\item 1995: ITMO
\item 1996: SPbSU
\item 1997: SPbSU
\item 1998: ITMO
\item 1999: ITMO
\item 2000: SPbSU
\item 2001: ITMO
\item 2002: ITMO
\item 2003: ITMO
\item 2004: ITMO
\item 2005: ITMO
\item 2006: PetrSU, ITMO
\item 2007: SPbSU
\item 2008: SPbSU
\item 2009: ITMO
\item 2010: ITMO
\item 2011: ITMO
\item 2012: ITMO
\item 2013: SPbSU
\item 2014: ITMO
\item 2015: ITMO
\item 2016: ITMO
\item 2017: ITMO
\item 2018: SPbSU
\item 2019: ITMO
\end{shortitems}

Help the contest archivist to remember the results of each contest and write a program that will read the year and print contest winners of that year in exactly the same format as above.

\InputFile
The only line of input contains a single integer $y$ ($1995 \le y \le 2019$), denoting the year. You don't need to process year numbers less than $1995$ or greater than $2019$, or incorrect year formats. It is guaranteed that you will be given a number between $1995$ and $2019$, inclusive.

\OutputFile
Print the winner of the contest in year $y$ exactly in the same format as in the list above.

\Examples

\begin{example}
\exmpfile{example.01}{example.01.a}%
\exmpfile{example.02}{example.02.a}%
\exmpfile{example.03}{example.03.a}%
\exmpfile{example.04}{example.04.a}%
\end{example}

\end{problem}

