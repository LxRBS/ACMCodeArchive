\problemname{ICPC Record Matching}

\par
When using ``software-as-a-service'' offerings, a user often has a
problem of matching records stored by the different services and
determining if they refer to the same entity (person, account,
order, etc.)  The ICPC is no exception.  While each participant has a
record in the central ICPC registration system, additional ``outside''
applications may be used to collect and process information for
functionality not provided by the central system.
\par
Once an ``outside'' application is used, it becomes necessary to match
the entries from both systems.  Unfortunately, in spite of careful
directions, sometimes it is not clear if records correspond to the
same person.  The primary sorts of mis-matches that occur are these:
\begin{enumerate}
	\item  E-mail addresses do not match.  This could be due to a misspelling,
	such as {\sl .eud} instead of {\sl .edu,} or different e-mail addresses
	that a participant used in the central ICPC system and the outside system.
	\item  Exact names do not match.  This could be due to a typing error, or
	varying use of legal names and nicknames.
	\par
\end{enumerate}
Your team is to write a program that will read lists of people from
the ICPC system and an outside system and determine which records in
each system do not match a record in the other system.  Two entries
are considered matched if {\sl either} the e-mail addresses are an exact
match {\sl or} the last name and first name are an exact match.

\section*{Input}

Input to your program is two lists of name and e-mail address
records.  Each record consists of a first name, a last name, and an
e-mail address, one per line, separated from each other by tabs.  The
first list is the records from the central ICPC registration system.
This list ends with an empty line.  The second list is the records
from the outside application.  The second list ends with the end-of-file.
(These lists are suitable test data, not actual ICPC data.)
\par
E-mail addresses do not exceed $64$ characters in length.
E-mail addresses consist of lower-case and upper-case English letters,
digits, and the special characters at-sign, underscore, hyphen, and
period.  E-mail addresses do not begin with special characters.
\par
First and last names do not exceed $24$ characters in length.  Names
consist of lower-case and upper-case English letters and hyphens.
Names do not begin with hyphens.
\par
Each input list
will contain at least $1$ but not more than $2\,000$ entries.  E-mail addresses
and (first name, last name) pairings will be unique within each list.

\section*{Output}

Your program is to print lines showing the
records from each list that could not be matched to the other list.
Your program is to first print the central ICPC registration records
that could not be matched, one per line.  Each line should consist of
the letter ``{\tt I}'', a single space, the e-mail address, a single space,
the last name, a single space, and the first name.  These are to be
printed in lexicographical e-mail address order.  The e-mail addresses, last
names, and first names are to be printed
exactly the way they appear in the input.  Once all such records are
printed, the outside application records that could not be matched
are to be printed the same way, except that each line should begin
with the letter ``{\tt O}''.
\par
Case is to be ignored for all record matching comparisons and sorting.
\par
Should all records from each system have a match in the other system,
your program is to print a line containing only the
string ``{\tt No mismatches.}''.
