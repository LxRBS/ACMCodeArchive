\documentclass[11pt,letterpaper]{article}
\usepackage{floatrow}
\usepackage{amsmath}
\usepackage{amsthm}
\usepackage{amssymb}
\usepackage{amsfonts}
\usepackage{color}
%\usepackage{hyperref}
\usepackage{algorithm}
\usepackage{caption}
\usepackage{MnSymbol}
\usepackage{url}
\usepackage[margin=1in]{geometry}
\usepackage[labelfont=bf,font=small,margin=1em]{caption} %font of captions
\usepackage{mathtools}
\usepackage{refcount}
\usepackage[shortlabels]{enumitem}

\usepackage{tikz}
\usetikzlibrary{shapes,snakes}
\usetikzlibrary{arrows.meta,automata,positioning,arrows}

\newcommand{\Z}{\mathbb{Z}}
\newcommand{\N}{\mathbb{N}}
\newcommand{\C}{\mathbb{C}}
\newcommand{\R}{\mathbb{R}}
\newcommand{\Q}{\mathbb{Q}}
\renewcommand{\Pr}[1]{{\mathop{\text{Pr}}\left[#1\right]}}
\newcommand{\Ex}[1]{{\mathop{\mathbb{E}}\left[#1\right]}}
\newcommand{\todo}[1]{{\bf \color{red} TODO: #1}}

\newtheorem*{claim}{Claim}
\newtheorem{theorem}{Theorem}
\newtheorem{lemma}[theorem]{Lemma}
\newtheorem{corollary}[theorem]{Corollary}
\newtheorem{conjecture}[theorem]{Conjecture}
\newtheorem{proposition}[theorem]{Proposition}
\newtheorem{definition}[theorem]{Definition}
\newtheorem{notation}[theorem]{Notation}
\newtheorem{example}[theorem]{Example}
\newtheorem{remark}[theorem]{Remark}
\newtheorem{problem}[theorem]{Problem}
\newtheorem{acknowledgment}[]{Acknowledgment}

\title{}
\author{}
\date{\today}

\begin{document}
\maketitle
%\illustration{.4}{Binomial_Trees.png}{Binomial trees of a binomial heap. Wikimedia, cc-by-sa}

Redundant binary notation is similar to binary notation, except instead of allowing only $0$’s and $1$’s for each digit, we allow any integer digit in the range $[0, t]$, where $t$ is some specified upper bound. For example, if $t = 2$, the digit $2$ is permitted, and we may write the decimal number $4$ as $100$, $20$, or $12$. If $t=1$, every number has precisely one representation, which is its typical binary representation. In general, if a number is written as $d_l d_{l-1} \ldots d_1 d_0$ in redundant binary notation, the equivalent decimal number is $d_l\cdot2^l + d_{l-1}\cdot2^{l-1} + \cdots + d_1\cdot2^1 + d_0\cdot2^0$.

Redundant binary notation can allow carryless arithmetic, and thus has applications in hardware design and even in the design of worst-case data structures. For example, consider insertion into a standard binomial heap. This operation takes $O(\log n)$ worst-case time but $O(1)$ amortized time. This is because the binary number representing the total number of elements in the heap can be incremented in $O(\log n)$ worst-case time and $O(1)$ amortized time. By using a redundant binary representation of the individual binomial trees in a binomial heap, it is possible to improve the worst-case insertion time of binomial heaps to $O(1)$.

However, none of that information is relevant to this question. In this question, your task is simple. Given a decimal number $N$ and the digit upper bound $t$, you are to count the number of possible representations $N$ has in redundant binary notation with each digit in range $[0, t]$ with no leading zeros.

\section*{Input}

Input consists of a single line with two decimal integers $N$ ($0 \leq N \leq 10^{16}$) and $t$ ($1 \leq t \leq 100$).


\section*{Output}

Output in decimal the number of representations the decimal number $N$ has in redundant binary notation with each digit in range $[0, t]$ with no leading zeros. Since the number of representations may be very large, output the answer modulo the large prime $998\,244\,353$.

\bibliographystyle{alpha}
\bibliography{ref}

\end{document}

