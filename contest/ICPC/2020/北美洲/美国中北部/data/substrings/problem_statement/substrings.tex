\input problem.tex
\def\probname{Substring Characters}
\def\probno{?}
\def\headeryear{2020/2021}
\fullheader
\par
The set of distinct characters in a string is referred to as the
generalized period of the string.  As an example, the generalized period
of the string ``{\tt aabbabb}'' is $\{\hbox {\rm `{\tt a}', `{\tt b}'}\}$.
\par
A proper substring is a contiguous substring that is contained in
a string and is not the string itself. So ``{\tt aabbabb}'' is not a
proper substring of the above example.
\par
A minimal proper substring is one that can have no character removed from
either end and still have the same generalized period. ``{\tt aabb}''
is a proper substring of the example, but it is not minimal. ``{\tt ab}''
is minimal.
\par
Unique means that multiple occurrences of the same minimal proper
substring in a string are only to be counted once. In the example,
``{\tt ab}'' appears twice, but is counted once---hence the number of
proper minimal unique substrings with the same generalized period of
the entire string is two: ``{\tt ab}'' and ``{\tt ba}''.
\par
Your team is to write a program to count the number of proper minimal
unique substrings of a given string that have the same generalized
period as the string itself.  Input to your program is a series of lines
terminated by end-of-file. Each line is a test case consisting of
alphanumeric characters (a--z, A--Z, 0--9).  Upper-case and lower-case
letters are distinct.  The new line character is not part of the test
case string.  No test case string will exceed 80 characters.
\par
For each input line print a line containing the number of proper minimal
unique substrings of the input string with no leading or trailing
whitespace and no extra leading signs or zeros.
\vskip 22 pt
\centerline{\it Sample Input}
\vskip 14 pt
\:aabbabb
\:abAB34aB3ba7
\:104001144
\:aaabcaaa
\:a
\:bb
\:bd
\:1234567
\vskip 22 pt
\centerline{\it Output for the Sample Input}
\vskip 14 pt
\:2
\:1
\:3
\:2
\:0
\:1
\:0
\:0
\vfill \eject
\end

