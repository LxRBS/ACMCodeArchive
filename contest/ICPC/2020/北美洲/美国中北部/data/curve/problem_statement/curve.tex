\input problem.tex
\input epsf.tex
\def\probname{Curve Speed}
\def\probno{3}
\def \headeryear {2020/2021}
\fullheader
\par
To help with vehicle stability, the outer edge of a road in a curve
is raised with respect to the inner edge. This is called super\-elevation
and is specified as the difference in elevation divided by the width 
of the road. It needs to be higher for faster speeds and sharper curves.
\par
The radius of a curve is the radius of the section of a circle along the
middle of the road where the curve is constant.  See Figure 1 for a
drawing of this.
\par
In some cases the curve may need a lower speed limit than straight
portions of the road. The super\-elevation shouldn't be more than about
.12 to keep vehicles from sliding off the road in slippery conditions.
\par
Your job is to calculate the maximum speed on a curve, given the radius
of the curve and the super\-elevation.
\par
The maximum speed is given by this formula:
$$V = \sqrt{(R*(S+.16))/.067}$$
where $V$ is the maximum speed in miles per hour, $R$ is the radius of
the curve in feet, and $S$ is the super\-elevation.
\par
The input is a series of lines terminated by end-of-file. Each line
will be a test case consisting of $R$ and $S$ separated by whitespace. $R$
will be an integer greater than 99 and less than 5281 and $S$ will be a 
real number greater than .009 and less than 1.0. Neither will have 
leading zeros.
\par
For each test case print a line containing the maximum speed rounded to
the nearest integer.  It is guaranteed the answer before rounding will
not be within $10^{-3}$ of a half-integer value.
\vskip \parskip
\line {\hfill \hbox {\epsfbox {F1CurveSpeed.eps}} \hfill}
\medskip
\centerline {{\bf Figure 1.} Section of a circle along the middle of a road with radius $R$.}
\vskip 22 pt
\centerline {\it Sample Input}
\vskip 14 pt
\:1433   .09
\:1433 .12
\:2000 .09
\:600 .12
\vfill \eject \continuedheader
\vskip \parskip
\centerline{\it Output for the Sample Input}
\vskip 14 pt
\:73
\:77
\:86
\:50
\vfill \eject
\end

