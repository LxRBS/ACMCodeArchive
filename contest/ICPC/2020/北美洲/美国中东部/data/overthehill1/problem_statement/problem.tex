\problemname{Over the Hill, Part 1}

Hill encryption (devised by mathematician Lester S.\ Hill in 1929) is a technique that makes use of matrices and modular arithmetic.  It is ideally used with an alphabet that has a prime number of characters, so we'll use the $37$ character alphabet {\tt A, B, $\ldots$, Z, 0, 1,
$\ldots$, 9,} and the space character.  The steps involved are the following:
\begin{enumerate}
\item Replace each character in the initial text (the {\em plaintext}) with the substitution {\tt A$\rightarrow 0$, B$\rightarrow 1$, $\ldots$, (space)$ \rightarrow 36$}.  If the plaintext is {\tt ATTACK AT DAWN} this becomes
\begin{center}
$0\ 19\ 19\ 0\ 2\ 10\ 36\ 0\ 19\ 36\ 3\ 0\ 22\ 13$
\end{center}
\item Group these number into three-component vectors, padding with spaces at the end if necessary.  After this step we have
\[
\left(
\begin{array}{c}
0 \\ 19 \\ 19
\end{array}
\right)
\left(
\begin{array}{c}
0 \\ 2 \\ 10
 \end{array}
\right)
\left(
\begin{array}{c}
36 \\ 0 \\ 19
\end{array}
\right)
\left(
\begin{array}{c}
36 \\ 3 \\ 0
\end{array}
\right)
\left(
\begin{array}{c}
22 \\ 13 \\ 36
\end{array}
\right)
\]
\item Multiply each of these vectors by a predetermined $3 \times 3$ encryption matrix using modulo $37$ arithmetic. If the encryption matrix is
\[
\left(
\begin{array}{ccc}
30 &  1 &  9 \\
 4 & 23 &  7 \\
 5 &  9 & 13
\end{array}
\right)
\]
then the first vector is transformed as follows:
\begin{eqnarray*}
\left(
\begin{array}{ccc}
30 &  1 &  9 \\
 4 & 23 &  7 \\
 5 &  9 & 13
\end{array}
\right)
\left(
\begin{array}{c}
0 \\ 19 \\ 19
\end{array}
\right)
& = &
\left(
\begin{array}{c}
%(30(0) + 1(19) + 9(19)) \mod 37 \\
%(4(0) + 23(19) + 7(19)) \mod 37 \\
%(5(0) + 9(19) + 13(19)) \mod 37
(30 \times 0 + 1 \times 19 + 9 \times 19) \mod 37 \\
(4 \times 0 + 23 \times 19 + 7 \times 19) \mod 37 \\
(5 \times 0 + 9 \times 19 + 13 \times 19) \mod 37
\end{array}
\right) \\
& = &
\left(
\begin{array}{c}
5 \\ 15 \\ 11
\end{array}
\right)
\end{eqnarray*}

\item After multiplying all the vectors by the encryption matrix, convert the resulting values back to the $37$-character alphabet and concatenate the results to obtain the encrypted {\em ciphertext}.  In our example the ciphertext is {\tt FPLSFA4SUK2W9K3}.
\end{enumerate}

This method can be generalized to work with any $n \times n$ encryption matrix in which case the initial plaintext is broken up into vectors of length $n$.  For this problem you will be given an encryption matrix and a plaintext and must compute
the corresponding ciphertext.

\section*{Input}
Input begins with a line containing a positive integer $n \leq 10$ indicating the size of the matrix and the vectors to use in the encryption.  After this are $n$ lines each containing $n$ non-negative integers specifying the encryption matrix. After this is a single line containing the plaintext consisting only of characters in the $37$-character alphabet specified above.

\section*{Output}
Output the corresponding ciphertext on a single line.
