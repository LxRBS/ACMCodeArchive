\problemname{New Maths}

\illustration{0.3}{carrylesssquareroot.png}{}

\noindent
``Drat!'' cursed Charles.  ``This stupid carry bar is not working
in my Engine!  I just tried to calculate the square of a number,
but it's wrong; all of the carries are lost.''

``Hmm,'' mused Ada, ``arithmetic without carries!  I wonder if
I can figure out what your original input was,
based on the result I see on the Engine.''

\emph{Carryless addition}, denoted by $\oplus$, is the same as normal addition, except any carries
are ignored (in base $10$). Thus, $37 \oplus 48$ is $75$, not $85$.

\emph{Carryless multiplication}, denoted by $\otimes$, is performed using the schoolboy algorithm
for multiplication, column by column, but the intermediate additions are
calculated using \emph{carryless addition}. More formally,
Let $a_m a_{m-1} \ldots a_1 a_0$ be the digits of $a$, where $a_0$ is its least significant digit.
Similarly define $b_n b_{n-1} \ldots b_1 b_0$ be the digits of $b$.
The digits of $c = a \otimes b$ are given by the following equation:
\[
c_k = \left( a_0 b_k \oplus a_1 b_{k-1} \oplus \cdots \oplus a_{k-1} b_1 \oplus a_k b_0 \right) \bmod{10},
\]
where any $a_i$ or $b_j$ is considered zero if $i > m$ or $j > n$. For example, $9 \otimes 1\,234$ is $9\,876$,
$90 \otimes 1\,234$ is $98\,760$, and $99 \otimes 1\,234$ is $97\,536$.


Given $N$, find the smallest positive integer $a$ such that $a \otimes a = N$.

\section*{Input}

The input consists of a single line with a positive integer $N$,
with at most $25$ digits and no leading zeros.

\section*{Output}

Print, on a single line, the least positive number $a$ such that $a \otimes a = N$.
If there is no such $a$, print `{\tt -1}' instead.

\section*{Examples}
