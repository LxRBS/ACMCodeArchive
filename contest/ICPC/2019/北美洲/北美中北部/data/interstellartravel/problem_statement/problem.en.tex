\problemname{Solar Energy}

\illustration{0.3}{interstellartravel.png}{}

\noindent
You are planning to travel in interstellar space in the hope of finding habitable
planets. You have already identified $N$ stars that
can recharge your spaceship via its solar panels. The only work left
is to decide the orientation of the spaceship that maximizes the
distance it can travel.

Space is modeled as a 2D plane, with the Earth at the origin.
The spaceship can be launched from the Earth in a straight line,
in any direction. Star $i$ can provide enough energy to travel $T_i$
distance if the spaceship is launched at an angle of $a_i$ with the
$x$-axis. If the angle is not perfectly aligned, then the spaceship
gets less energy. Specifically, if the launch direction makes
an angle of $a$ with the $x$-axis, then it gets enough energy to travel
distance of \[\max(0, T_i - s_i \cdot \mathrm{dist}(a_i, a))\] from star $i$, where
$\mathrm{dist}(a,b)$ is the minimum radians needed to go from angle $a$ to $b$. The
distance that the spaceship can travel is simply the sum of the
distances that each star contributes. Find the maximum distance $T$
that the starship can travel.

\section*{Input}

The first line contains the value $N$, $1\le N\le 10^5$.  Following
this are $N$ lines each containing three real numbers
$T_i$, $s_i$, and $a_i$, with $0<T_i\le 1\,000$, $0\le s_i\le 100$, and
$0\le a_i < 2\pi$. All real numbers in the input have at most $6$ digits after
the decimal point.

\section*{Output}

On a single line output the maximum distance the spacecraft can
travel. Your answer is considered correct if it has an absolute or relative
error of at most $10^{-6}$.

\section*{Examples}
