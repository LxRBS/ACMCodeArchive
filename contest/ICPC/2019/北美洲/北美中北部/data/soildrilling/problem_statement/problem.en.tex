\problemname{Weird Flecks, But OK}

\illustration{0.25}{soildrilling/problem_statement/drill-bit-vector-clipart.png}{}
% From https://www.goodfreephotos.com/vector-images/drill-bit-vector-clipart.png.php

An artist who wanted to create an installation where his works
appeared to be floating in midair has cast a large cube of clear
acrylic to serve as a base. Unfortunately, during the casting, some
small flecks of dirt got into the mix, and now appear as a cluster of
pinpoint flaws in the otherwise clear cube.

He wants to drill out the portion of the cube containing the flaws so
that he can plug the removed volume with new, clear acrylic. He would
prefer to do this in one drilling step. For stability's sake, the drill must
enter the cube perpendicular to one of its faces.

Given the $(x,y,z)$ positions of the flaws, and treating the size of the
flaws as negligible, what is the smallest diameter drill bit that can
be used to remove the flaws in one operation??

The drill may enter any one of the cube faces, but must be positioned
orthogonally to the face. 

\section*{Input}

The first line of input will contain an integer $N$ denoting the
number of flaws. $3 \leq N \leq 5\,000$

This is followed by $N$ lines of input, each containing three real
numbers in the range $-1\,000.0\ldots 1\,000.0$, denoting the
$(x,y,z)$ coordinates of a single flaw. Each number will contain at
most $6$ digits following a decimal point. The decimal point may be
omitted if all succeeding digits are zero.

\section*{Output}

Print the diameter of the smallest drill bit that would remove all the flaws.

The answer is considered correct if the absolute or relative error is
less than $10^{-4}$
