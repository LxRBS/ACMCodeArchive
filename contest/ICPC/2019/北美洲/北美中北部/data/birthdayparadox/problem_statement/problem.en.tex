\problemname{Birthday Paradox}

\illustration{0.4}{birthday_plot.png}{The probability of $n$ unique birthdays among $n$ people.}

The Birthday Paradox is the name given to the surprising fact that if there are
just $23$ people in a group, there is a greater than $50\%$ chance that a pair
of them share the same birthday. The underlying assumptions for this are that
all birthdays are equally likely (which isn't quite true), the year has exactly
$365$ days (which also isn't true), and the people in the group are uniformly
randomly selected (which is a somewhat strange premise).  For this problem,
we'll accept these assumptions.

Consider what we might observe if we randomly select groups of $P=10$ people.
Once we have chosen a group, we break them up into subgroups based on shared birthdays.
Among \textit{many} other possibilities, we might observe the following
distributions of shared birthdays:
\begin{itemize}
    \item all $10$ have different birthdays, or
    \item all $10$ have the same birthday, or
    \item $3$ people have the same birthday, $2$ other people
        have the same birthday (on a different day), and the remaining $5$ all
        have different birthdays.
\end{itemize}
Of course, these distributions have different probabilities of occurring.

Your job is to calculate this probability for a given distribution of people
sharing birthdays.  That is, if there are $P$ people in a group, how probable
is the given distribution of shared birthdays (among all possible distributions
for $P$ people chosen uniformly at random)?

\section*{Input}

The first line gives a number $n$ where $1 \le n \le 365$.
The second line contain integers $c_1$ through $c_n$, where
$1 \le c_i \le 100$ for all $c_i$. The value $c_i$ represents the number of
people who share a certain birthday (and whose birthday is distinct from the
birthdays of everyone else in the group).

\section*{Output}

Compute the probability $b$ of observing a group of people with the given distribution of
shared birthdays. Since $b$ may be quite small, output instead $\log_{10}(b)$.
Your submission's answer is considered correct if it has an absolute or
relative error of at most $10^{-6}$ from the judge's answer.

\section*{Explanations}

The first sample case shows $P=2$ people with distinct birthdays. The
probability of this occurring is $b = 364/365 \approx 0.9972602740$,
and $\log_{10}(b) \approx -0.001191480807419$.

The second sample case represents the third example in the list given earlier
with $P=10$ people. In this case, the probability is $b \approx 0.0000489086$,
and $\log_{10}(b) \approx -4.310614508857128$.
