\problemname{Euler's Number}

Euler’s number (you may know it better as just $e$) has a special place in mathematics. You may have encountered $e$ in calculus or economics (for computing compound interest), or perhaps as the base of the natural logarithm, $\ln{x}$, on your calculator.

While $e$ can be calculated as a limit, there is a good approximation that can be made using discrete mathematics. The formula for $e$ is:
\begin{align*}
e &= \displaystyle\sum_{i=0}^n\dfrac{1}{i!}\\
&= \dfrac{1}{0!} +\dfrac{1}{1!} +\dfrac{1}{2!}+\dfrac{1}{3!}+\dfrac{1}{4!} + \cdots\\
%&= \dfrac{1}{1} + \dfrac{1}{1} + \dfrac{1}{1\cdot2} + \dfrac{1}{1\cdot2\cdot3} + \dfrac{1}{1\cdot2\cdot3\cdot4} + \cdots\\
\end{align*}
Note that $0! = 1$. Now as $n$ approaches $\infty$, the series converges to $e$. When $n$ is any positive constant, the formula serves as an approximation of the actual value of $e$.  (For example, at $n=10$ the approximation is already accurate to $7$ decimals.)

You will be given a single input, a value of $n$, and your job is to compute the approximation of $e$ for that value of $n$.

\section*{Input}
A single integer $n$, ranging from $0$ to $10\,000$.

\section*{Output}
A single real number -- the approximation of $e$ computed by the formula with the given $n$. All output must be accurate to an absolute or relative error of at most $10^{-12}$.
