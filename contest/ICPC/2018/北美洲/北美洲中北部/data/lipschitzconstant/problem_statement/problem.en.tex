\problemname{Lipschitz Constant}

Today you are doing your calculus homework, and you are tasked with finding a Lipschitz constant for a function \textit{f(x)}, which
is defined for $N$ integer numbers $x$ and produces real values.
Formally, the Lipschitz constant for a function \textit{f} is the smallest real number $L$ such that for any $x$ and $y$ with \textit{f(x)} and \textit{f(y)} defined we have:
$$|f(x) - f(y)| \leq L \cdot |x - y|.$$

\section*{Input}
The first line contains $N$ -- the number of points for which \textit{f} is defined. The next $N$ lines each
contain an integer $x$ and a real number $z$, which mean that $f(x) = z$. Input satisfies the following constraints:
\begin{itemize}
	\item $2 \leq N \leq 200\,000$.
	\item All $x$ and $z$ are in the range $-10^9 \leq x,z \leq 10^9$.
%	\item For each $x,z$ such that $f(x) = z$ we have $-10^9 \leq x \leq 10^9$ and $-10^9 \leq z \leq 10^9$.
	\item All $x$ in the input are distinct.
\end{itemize}


\section*{Output}
Print one number -- the Lipschitz constant. The result will be considered correct if it is within an absolute error of $10^{-4}$ from the jury's answer.
