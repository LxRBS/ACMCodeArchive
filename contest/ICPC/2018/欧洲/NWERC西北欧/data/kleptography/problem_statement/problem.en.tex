\problemname{Kleptography}
John likes simple ciphers. He had been using the ``Caesar'' cipher to encrypt
his diary until recently, when he learned a hard lesson about its strength by
catching his sister Mary browsing through the diary without any problems.

Rapidly searching for an alternative, John found a solution: the famous ``Autokey''
cipher. He uses a version that takes the $26$ lower-case letters
`\texttt{a}'--`\texttt{z}' and internally translates them in alphabetical order to the numbers
$0$ to $25$.

The encryption key $k$ begins with a secret prefix of $n$ letters. Each
of the remaining letters of the key is copied from the letters of the
plaintext $a$, so that $k_{n+i} = a_{i}$ for $i \geq 1$. Encryption of
the plaintext $a$ to the ciphertext $b$ follows the formula
$b_i = a_i + k_i \bmod 26$.

Mary is not easily discouraged. She was able to get a peek at the last $n$
letters John typed into his diary on the family computer before he noticed her,
quickly encrypted the text document with a click, and left. This could be her
chance.

\section*{Input}
The input consists of:
\begin{itemize}
\item One line with two integers $n$ and $m$
      ($1 \le n \le 30$, $n + 1 \le m \le 100$),
where $n$ is the length of the keyword as well as the number of
letters Mary saw, and $m$ is the length of the text.
\item One line with $n$ lower-case letters, the last $n$ letters of the plaintext.
\item One line with $m$ lower-case letters, the whole ciphertext.
\end{itemize}

\section*{Output}

Output the plaintext of John's diary.
