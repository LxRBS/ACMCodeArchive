\begin{problem}{Interactive Array Guessing}{standard input}{standard output}{3 seconds}{512 megabytes}

This is an interactive problem.

Jury has prepared $n$ non-empty integer arrays $a_1, a_2, \ldots, a_n$. Each array consists of at most 10 positive integers not exceeding 1000. The elements of each array are pairwise distinct.

You can make queries of the following kind: choose several indices $q_1, q_2, \ldots, q_m$, where $1 \le m \le n$ and $1 \le q_i \le n$. These indices do not have to be distinct. The testing system tells you the contents of arrays $a_{q_1}, a_{q_2}, \ldots, a_{q_m}$ in the same order. However, these contents are concatenated without any delimeters.

Your task is to find the contents of all $n$ arrays.

\Interaction
First, the testing system writes the integer $n$~--- the number of arrays ($1 \leq n \leq 1000$).

Your solution shall print requests of two types:
\begin{itemize}
\item ``\t{? }$m\ q_1\ q_2\ \ldots\ q_m$'' corresponds to a query about the contents of arrays $a_{q_1}, a_{q_2}, \ldots, a_{q_m}$. The testing system responds with the total length of these arrays, followed by the contents of array $a_{q_1}$, followed by the contents of array $a_{q_2}$, $\ldots$, followed by the contents of array $a_{q_m}$.
\item ``\t{! }$k_1\ a_{1, 1}\ \ldots\ a_{1, k_1}\ \ldots\ k_n\ a_{n, 1}\ \ldots\ a_{n, k_n}$'' tells that your program has determined the contents of all the arrays, where $k_i$ is the length of array $a_i$. The request is formed by the length of array $a_1$, followed by the contents of array $a_1$, followed by the length of array $a_2$, followed by the contents of array $a_2$, $\ldots$, followed by the length of array $a_n$, followed by the contents of array $a_n$.
\end{itemize}

Don't forget to flush the output after each request!

Your solution must issue exactly one request of the second type. It must be the last request, and the solution must terminate gracefully after issuing it.

Your solution is allowed to issue at most 20 requests of the first type.




\Example

\begin{example}
\exmp{3
~
5 1 1 2 2 1
~
4 1 2 1 1
~
2 1 2}{~
? 3 1 2 3
~
? 3 1 3 1
~
? 1 2
~
! 1 1 2 1 2 2 2 1}%
\end{example}

\end{problem}

